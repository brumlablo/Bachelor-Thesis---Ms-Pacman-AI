\chapter{Obsah CD}
Praktická část práce je uložena na CD. Obsah CD a jeho popis:
\begin{itemize}
\item \texttt{doc} - dokumentace k práci \texttt{projekt.pdf} a veškeré soubory k jejímu sestavení, výsledné \textit{pdf} lze znovu sestavit příkazem \texttt{make}
\item \texttt{src} - zdrojové kódy frameworku s (včetně složky \texttt{layouts} s mapami pro \texttt{pacman.py}) implementacemi agentů v souborech:
\begin{itemize}
\item \texttt{multiAgents.py} - implementace chování základních agentů
\item \texttt{valueiterationAgents.py} - implementace chování \texttt{valueiterationAgent}
\item \texttt{qlearningAgents.py} - implementace chování \texttt{QLearningAgent}, \texttt{PacmanQAgent} a \texttt{ApproximateQAgent}
\end{itemize}
\end{itemize}

\chapter{Manuál}
\label{priloha:manual}
\section{\texttt{pacman.py}}
Pro demo Ms. Pacman byly používány základní parametry:
\begin{itemize}
\item \texttt{-h} pro výpis celé nápovědy a všech možných parametrů
\item \texttt{-m} manuální režim (vyzkoušení grafického rozhraní dema)
\item \texttt{--frameTime 0} bez animace
\item \texttt{-q} minimalistický režim bez grafického rozhraní
\item \texttt{-t} textový režim bez grafického rozhraní 
\item \texttt{-f} fixed random seed (pro fixní modelaci náhodnosti scénářů)
\item \texttt{-n} počet her celkově
\item \texttt{-x} počet tréninkových epizod
\item \texttt{-l} druh mapy (ze složky \texttt{src/layouts}) - např. \texttt{smallClassic}, \texttt{mediumClassic}, \newline \texttt{minimaxClassic}, \texttt{openClassic},...
\item \texttt{-p} druh agenta - např. \texttt{ReflexAgent}, \texttt{ExpectimaxAgent}, \newline \texttt{PacmanQAgent}, \texttt{ApproximateQAgent},...
\item \texttt{-a arg1=hodnota,arg2=hodnota} agentovy další argumenty,\newline např. hloubka: \texttt{-a depth=3}
\end{itemize}
\subsection{Příklady použití}
\texttt{ReflexAgent}, 2 nepřátelé, defaultní mapa
\begin{lstlisting}[language=Python,texcl=true]
python pacman.py -p ReflexAgent -k 2
\end{lstlisting}
\texttt{ExpectimaxAgent}, 5 her se základním výpisem bez GUI, hloubka 3
\begin{lstlisting}[language=Python,texcl=true]
python pacman.py -p ExpectimaxAgent -l smallClassic -a depth=3,
-n 5 -q
\end{lstlisting}
\texttt{PacmanQAgent}, 2000 epizod tréninku, 10 zobrazených epizod provádení optimální strategie, na mapě \texttt{smallGrid}
\begin{lstlisting}[language=Python,texcl=true]
python pacman.py -p PacmanQAgent -x 2000 -n 2010 -l smallGrid
\end{lstlisting}
\texttt{PacmanQAgent}, 10 epizod tréninku na mapě \texttt{smallGrid}
\begin{lstlisting}[language=Python,texcl=true]
python pacman.py -p PacmanQAgent -n 10 -l smallGrid -a
numTraining=10
\end{lstlisting}
\texttt{ApproximateQAgent} 60 epizod z nichž 50 trénink (10 zobrazených epizod provádení optimální strategie), na mapě \texttt{mediumGrid}, použitý lepší Extractor vlastností
\begin{lstlisting}[language=Python,texcl=true]
python pacman.py -p ApproximateQAgent -a extractor=BetterExtractor
-n 60 -x 50 -l mediumClassic
\end{lstlisting}

\section{\texttt{gridworld.py}}
Pro vizualizaci hodnot pokročilých agentů - především agenta \texttt{ValueIterationAgent} byly používány základní parametry:
\begin{itemize}
\item \texttt{-h} pro výpis celé nápovědy a všech možných parametrů
\item \texttt{-q} minimalistický režim bez grafického rozhraní
\item \texttt{-t} textový režim bez grafického rozhraní
\item \texttt{-s} nastavení rychlosti
\item \texttt{-k} počet epizod
\item \texttt{-v} zobrazení hodnot po každé iteraci
\item \texttt{-g} druh gridworld problému - např. \texttt{BookGrid}, \texttt{BridgeGrid}, \texttt{DiscountGrid},...
\item \texttt{-a} druh agenta - \texttt{value} pro Value Iteration, \texttt{q} pro Q-Learning
\item \texttt{-d} agentovy další argumenty např. hloubka: \texttt{-a depth=3}
\item \texttt{-r} \texttt{livingReward}, odměna přechodu ze stavu do neterminálního následníka
\item \texttt{-n} \texttt{noise}, pravděpodobnost přechodu 
\item \texttt{-e} \texttt{epsilon}, náhodnost akcí při tréninku
\item \texttt{-l} \texttt{alpha}, rychlost TD učení
\end{itemize}
\subsection{Příklady použití}
\texttt{ValueIterationAgent}, defaultní mapa, 100 iterací, 10 epizod
\begin{lstlisting}[language=Python,texcl=true]
python gridworld.py -a value -i 100 -k 10
\end{lstlisting}
\texttt{ValueIterationAgent}, typ gridworld problému \texttt{DiscountGrid}, 5 iterací, 10 defaultních epizod
\begin{lstlisting}[language=Python,texcl=true]
python gridworld.py -a value -i 5 -g DiscountGrid
\end{lstlisting}
\texttt{ValueIterationAgent}, typ gridworld problému \texttt{BridgeGrid}, 100 iterací, \texttt{gamma} = 0.9, \texttt{noise} =  0.2
\begin{lstlisting}[language=Python,texcl=true]
python gridworld.py -a value -i 100 -g BridgeGrid --discount 0.9
--noise 0.2
\end{lstlisting}
\texttt{QLearningAgent}, 5 epizod tréninku, manuálně
\begin{lstlisting}[language=Python,texcl=true]
python gridworld.py -a q -k 5 -m
\end{lstlisting}
\texttt{QLearningAgent}, 20 epizod tréninku na mapě \texttt{DiscountGrid},  \texttt{alpha} = 0.8, \texttt{epsilon} = 0.5
\begin{lstlisting}[language=Python,texcl=true]
python gridworld.py -a q -k 20 -g DiscountGrid --learningRate 0.8
--epsilon 0.5
\end{lstlisting}
%\chapter{Konfigrační soubor}
%\chapter{RelaxNG Schéma konfiguračního soboru}
%\chapter{Plakat}

